\documentclass{article}
% Package setup
% HyperRef
\usepackage{color}   %May be necessary if you want to color links
\usepackage{hyperref}
\hypersetup{
    colorlinks=true, %set true if you want colored links
    linktoc=all,
    linkcolor=black,
    urlcolor=blue
}

\begin{document}
% Top matter
\title{CGEN LLVM-IR \\ Design Document \vfill}
\author{Leonardo Arcari \\ Politecnico di Milano}
\date{February 2018}
\maketitle
\thispagestyle{empty}
\clearpage

% Table of contents
\tableofcontents
\clearpage

% Introductory section
\pagenumbering{arabic}
\section{Introduction}
\subsection{Scope}
This document is meant to provide a resource to those who are going to work with GNU CGEN and my extension to it: CGEN LLVM-IR. The purpose of this paper is to introduce the reader first to GNU CGEN from a code perspective, as GNU CGEN already provides a user guide. The reader will find in this document a code analysis, with a, possibly more clear, description of the main classes in Scheme source code in order to use them effectively.

In second place, I will provide a similar description of the code that I wrote in order to extend GNU CGEN to allow the generation of C++ programs capable of translating binary programs into a semantically equivalent representation in LLVM-IR language.

\subsection{Out of scope}
In this paper I am not going to describe several topics related to GNU CGEN
\begin{itemize}
\item How to run GNU CGEN. There is a manual online for it.\footnote{\url{https://sourceware.org/cgen/docs/cgen_2.html}}
\item What are the features of GNU CGEN. There is a manual online for it.\footnote{\url{https://sourceware.org/cgen/docs/cgen_1.html}}
\item What is CGEN RTL and what each language feature does. There is a manual online for it.\footnote{\url{https://sourceware.org/cgen/docs/cgen_3.html}}
\item How to write a CGEN application to define your CPU architecture in RTL. Guess what? There's a manual online for it.\footnote{\url{https://sourceware.org/cgen/docs/cgen_8.html}}
\end{itemize}

\subsection{Project History}
CGEN LLVM-IR generator is part of the project I was assigned to while taking the \emph{Code Transformation and Optimization} course held by Professor G. Agosta\footnote{\url{https://home.deib.polimi.it/agosta}} in the A.Y. 2017/2018. The idea of extending GNU CGEN, in order to generate C++ translators capable of producing a semantically-equivalent representation in LLVM-IR of a binary for a given architecture, is from Alessandro Di Federico, PhD\footnote{\url{https://clearmind.me/}}. 

\clearpage
\section{GNU CGEN}
\subsection{Introduction to CGEN}
\subsection{CGEN RTL classes}
\subsection{Code Analysis}
\subsubsection{Entry Point}
\subsubsection{RTL-C Generator}

\clearpage
\section{CGEN LLVM-IR}
\subsection{CGEN-IR common}
\subsection{IR-Gen registers}
\subsection{IR-Gen decoder}
\subsection{RTL-CPP Generator}
\end{document}
